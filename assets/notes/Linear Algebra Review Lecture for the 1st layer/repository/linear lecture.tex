\documentclass[lang=cn,newtx,10pt,scheme=chinese]{elegantbook}

\title{线性代数辅导讲义}
\subtitle{一层次}
\author{Zebray}
\extrainfo{Attention is all you need.}

\setcounter{tocdepth}{3}

\cover{cover.jpg}

% 本文档命令
\usepackage{array}
\newcommand{\ccr}[1]{\makecell{{\color{#1}\rule{1cm}{1cm}}}}

% 修改标题页的橙色带
\definecolor{customcolor}{RGB}{32,178,170}
\colorlet{coverlinecolor}{customcolor}
\usepackage{cprotect}

\addbibresource[location=local]{reference.bib} % 参考文献,不要删除

\usepackage{rotating}

\begin{document}

\maketitle
\frontmatter

\pagestyle{empty} % 删除页码

\tableofcontents

\mainmatter

\chapter{行列式计算}

\begin{introduction}[需要熟悉的知识点]
	\item 2阶、3阶行列式手算
	\item 三角阵和反三角阵的行列式
	\item 分块三角阵和分块反三角阵的行列式
	\item 3种初等变换及对应行列式变化
	\item 按某行或某列展开的Laplace公式
	\item Vandemonde行列式
	\item 余子式$M_{ij}$与代数余子式$A_{ij}$定义及性质
	\item $\left|AB\right| = \left|A\right| \left|B\right|$
\end{introduction}

\section{直接计算}

\subsection{初等变换}
\par
\textbf{务必多加练习!}
\par
一般步骤:
\begin{itemize}
  	\item 通过初等变换
  	\item 得到某一行(列)全为1
		\begin{itemize}
			\item 某一行(列)全相等亦可,只需提出即得到全1
		\end{itemize}
    \item 再次初等变换得到某一行(列)只有一个1
    \item Laplace公式
\end{itemize}
\par
常见特征:
\begin{itemize}
	\item 每行(列)和全相等
	\item 每行(列)与其相邻行(列)有明显规律
	\item 某行(列)只有个别项不是0
\end{itemize}

\begin{exercise}
  (2015W-1-1) 设$A=(a_{ij})_{n \times n}, a_{ij} = ij$,求$\left| A \right|.$
\end{exercise}

\begin{exercise}
  (2017S-5) 计算行列式$
  D_n = 
  \left|
	\begin{array}{cccccc}
		b       & b       & \cdots & b      & b       & a       \\
		b       & b       & \cdots & b      & a       & b       \\
		b       & b       & \cdots & a      & b       & b       \\
		\vdots  & \vdots  &        & \vdots & \vdots  & \vdots  \\
		b       & a       & \cdots & b      & b       & b       \\
		a       & b       & \cdots & b      & b       & b       \\
	\end{array}
  \right|
  .
  $
\end{exercise}

\begin{exercise}
  (2018W-1-5) 计算行列式$
  D_n = 
  \left|
	\begin{array}{ccccc}
		1       & 2       & 3       & \cdots  & n       \\
		2       & 3       & 4       & \cdots  & n-1     \\
		3       & 4       & 5       & \cdots  & n-2     \\
		\vdots  & \vdots  & \vdots  &         & \vdots  \\
		n       & n-1     & n-2     & \cdots  & 1       \\
	\end{array}
  \right|
  .
  $
\end{exercise}

\begin{exercise}
  (2019S-4) 计算$f(\pi)$与$f'(\pi)$,此处
  $$
    f(x) =
    \left|
	\begin{array}{ccc}
		a_1 & b_1 & a_1 x^2 + b_1 x + c_1 \\
		a_2 & b_2 & a_2 x^2 + b_2 x + c_2 \\
		a_3 & b_3 & a_3 x^2 + b_3 x + c_3 \\
	\end{array}
    \right|
	.
  $$
\end{exercise}

\begin{exercise}
  (2019W-1-1) 计算行列式$D = 
  \left|
	\begin{array}{cccc}
		2   & 3   & 4   & 5 \\
		1   & 2   & 3   & 4 \\
		-1  & 3   & 1   & 0 \\
		1   & 2   & 1   & 3 \\
	\end{array}
  \right|
  .
  $
\end{exercise}

\begin{exercise}
  (2020S-1-1)  计算行列式$D = 
  \left|
	\begin{array}{cccc}
		1   & 2   & 3   & 4 \\
		-1  & 1   & 2   & 3 \\
		0   & -1  & 1   & 2 \\
		0   & 0   & -1  & 1 \\
	\end{array}
  \right|
  .
  $
\end{exercise}

\subsection{可递推型}
常见特征为使用Laplace公式将行列式降阶后变为所求行列式的低阶版本或一个局部。
\begin{itemize}
	\item 三对角阵或反三对角阵
	\item 中心对称且边界稀疏
\end{itemize}
递推式通过Laplace公式得到。

\begin{example}
	(2016W-1-5)计算行列式$
	D_n = 
	\left|
	\begin{array}{cccccc}
		0       & 0       & \cdots & 0      & 1       & 2       \\
		0       & 0       & \cdots & 1      & 2       & 1       \\
		0       & 0       & \cdots & 2      & 1       & 0       \\
		\vdots  & \vdots  &        & \vdots & \vdots  & \vdots  \\
		1       & 2       & \cdots & 0      & 0       & 0       \\
		2       & 1       & \cdots & 0      & 0       & 0       \\
	\end{array}
	\right|
	.
	$
\end{example}

\begin{solution}
	\begin{align*}
		D_n &= 
		(-1)^{n+1} \times 2
		\left|
		\begin{array}{ccccc}
			0       & 0       & \cdots & 1      & 2         \\
			0       & 0       & \cdots & 2      & 1         \\
			\vdots  & \vdots  &        & \vdots & \vdots	\\
			1       & 2       & \cdots & 0      & 0         \\
			2       & 1       & \cdots & 0      & 0         \\
		\end{array}
		\right|
		+ (-1)^{n+2}
		\left|
		\begin{array}{ccccc}
			0       & 0       & \cdots & 0      & 1         \\
			0       & 0       & \cdots & 2      & 1         \\
			\vdots  & \vdots  &        & \vdots & \vdots	\\
			1       & 2       & \cdots & 0      & 0         \\
			2       & 1       & \cdots & 0      & 0         \\
		\end{array}
		\right| \\
		&= (-1)^{n+1} \times 2D_{n-1} + (-1)^n \times (-1)^n
		\left|
		\begin{array}{cccc}
			0       & 0       & \cdots & 2 		\\
			\vdots  & \vdots  &        & \vdots	\\
			1       & 2       & \cdots & 0      \\
			2       & 1       & \cdots & 0      \\
		\end{array}
		\right| \\
		&= (-1)^{n+1} \times 2D_{n-1} + D_{n-2}
	\end{align*}
	设
	$$
	D_n + a D_{n-1} = b(D_{n-1} + a D_{n-2})
	$$
	即
	$$
	D_n = (b - a)D_{n-1} + ab D_{n-2} = (-1)^{n+1} \times 2D_{n-1} + D_{n-2}
	$$
	从而
	$
	\begin{cases}
		b - a = (-1)^{n+1} \times 2 \\
		ab = 1
	\end{cases}
	$
	,得
	$
	\begin{cases}
		a = (-1)^{n} \\
		b = (-1)^{n+1}
	\end{cases}
	$
	,结合$D_1=2,D_2 = \left|
	\begin{array}{cc}
		1 & 2 \\
		2 & 1 \\
	\end{array}
	\right|
	= -3$
	\begin{align*}
		D_n + (-1)^{n} D_{n-1} &= (-1)^{n+1}(D_{n-1} + (-1)^{n} D_{n-2}) \\
		&= (-1)^{\sum_{k=4}^{n+1}k}(D_2 + (-1)^3 D_1) \\
		&= (-1)^{\frac{(n+5)(n-2)}{2}}(D_2 - D_1) \\
		&= (-1)^{\frac{n^2 + 3n - 10}{2}}(-5) \\
		&= (-1)^{\frac{n(n+1)}{2}}5 \\
	\end{align*}
	对$n$奇偶分类即可。
\end{solution}

\begin{exercise}
	计算行列式$D_{n} = 
	\left|
	\begin{array}{cccc}
		\alpha + \beta	& \alpha\beta	&     		&					\\
	  	1				& \ddots		& \ddots	&					\\
	  	    			& \ddots		& \ddots	& \alpha\beta		\\
	  	    			&    			& 1  		& \alpha + \beta	\\
	\end{array}
	\right|
	.
	$
\end{exercise}

\begin{exercise}
	计算行列式$D_{2n} = 
	\left|
	\begin{array}{cccccc}
		a   &    										&     	&   & 											& b	\\
	  		& \ddots   									&     	&   & \begin{rotate}{90}$\ddots$\end{rotate}	&   \\
	  	    &   										& a		& b &											&	\\
	  	    &    										& b  	& a &											&	\\
		    & \begin{rotate}{90}$\ddots$\end{rotate}	&    	&   & \ddots									&	\\
		b   &    										&    	&   &											& a	\\
	\end{array}
	\right|
	.
	$
\end{exercise}

\begin{exercise}
	计算行列式$D_{n+1} = 
	\left|
	\begin{array}{cccccc}
		a_0 	& a_1		& a_2 		& \cdots	& a_{n-1}	& a_n		\\
	  	-1		& x			& 0    		& \cdots	& 0			& 0  		\\
	  	0		& -1		& x			& \cdots	& 0			& 0			\\
	  	\vdots	& \vdots 	& \vdots  	&			& \vdots	& \vdots 	\\
		0   	& 0			& 0			& \cdots	& -1		& x			\\
	\end{array}
	\right|
	.
	$
\end{exercise}

\subsection{Vandemonde}

$$
\left|
\begin{array}{ccccc}
    1         & 1         & \cdots & 1         \\
    x_1       & x_2       & \cdots & x_n       \\
    x_1^2     & x_2^2     & \cdots & x_n^2     \\
    \vdots    & \vdots    &        & \vdots    \\
    x_1^{n-2} & x_2^{n-2} & \cdots & x_n^{n-2} \\
    x_1^{n-1} & x_2^{n-1} & \cdots & x_n^{n-1} \\
\end{array}
\right|
=
\prod_{1 \le j < i \le n}(x_i - x_j)
$$
\begin{itemize}
    \item $j < i$而非$j \le i$
    \item $x_i - x_j$而非$x_j - x_i$
\end{itemize}

\begin{example}
  (2015W-2-(1)) \\
  (1) 计算$D_{n+1} = 
  \left|
    \begin{array}{cccccc}
        1         & 1         & \cdots & 1          & 1       \\
        x_1       & x_2       & \cdots & x_n        & x       \\
        x_1^2     & x_2^2     & \cdots & x_n^2      & x^2     \\
        \vdots    & \vdots    &        & \vdots     & \vdots  \\
        x_1^{n-2} & x_2^{n-2} & \cdots & x_n^{n-2}  & x^{n-2} \\
        x_1^{n-1} & x_2^{n-1} & \cdots & x_n^{n-1}  & x^{n-1} \\
        x_1^n     & x_2^n     & \cdots & x_n^n      & x^n     \\
    \end{array}
  \right|
  .
  $
\end{example}

\begin{solution}
    \ \\
    (1) $D_{n+1} = \prod_{1 \le j < i \le n}(x_i - x_j) \prod_{i=1}^n(x - x_i)$
\end{solution}

\subsection{秩一修正型}

对于可逆的方阵$A$,
$$
\left|A + \alpha\beta^\mathrm{T}\right| = (1+\beta^\mathrm{T}A^{-1}\alpha)\left|A\right|
$$
\begin{proof}
    注意到
	$$
	\left|
    \begin{array}{cc}
        A + \alpha\beta^\mathrm{T}	& 0	\\
        -\beta^\mathrm{T}       	& 1	\\
    \end{array}
  	\right|
	=
	\left|
    \begin{array}{cc}
        I	& -\alpha	\\
        0	& 1			\\
    \end{array}
  	\right|
	\left|
    \begin{array}{cc}
        A					& \alpha	\\
        -\beta^\mathrm{T}	& 1			\\
    \end{array}
  	\right|
	=
	\left|
    \begin{array}{cc}
        I						& \alpha	\\
        \beta^\mathrm{T} A^{-1}	& 1			\\
    \end{array}
  	\right|
	\left|
    \begin{array}{cc}
        A					& \alpha	\\
        -\beta^\mathrm{T}	& 1			\\
    \end{array}
  	\right|
	=
	\left|
    \begin{array}{cc}
        A	& \alpha							\\
        0	& 1+\beta^\mathrm{T}A^{-1}\alpha	\\
    \end{array}
  	\right|
	$$
\end{proof}

\begin{exercise}
    计算$n$阶行列式
	$$
	D_n = 
    \left|
    \begin{array}{ccccc}
        a_1 + x_1	& a_2		& \cdots	& a_n		\\
        a_1   		& a_2 + x_2 & \cdots	& a_n		\\
        \vdots		& \vdots	& 			& \vdots	\\
        a_1   		& a_2		& \cdots	& a_n + x_n \\
    \end{array}
    \right|
    ,$$
    其中$x_i \neq 0,i = 1, 2, \dots, n.$
\end{exercise}

\section{余子式与代数余子式之和类型}

\begin{itemize}
    \item $M_{ij} = (-1)^{i+j}A_{ij}$
    \item 
    \begin{align*}
        D &= \sum_{i = 1}^n a_{ij}A_{ij},\forall j = 1,2,\dots,n \\
        &= \sum_{j = 1}^n a_{ij}A_{ij},\forall i = 1,2,\dots,n
    \end{align*}
    \item \begin{align*}
        0 &= \sum_{i = 1}^n a_{ij_0}A_{ij},\forall j = 1,2,\dots,n,j \neq j_0 \\
        &= \sum_{j = 1}^n a_{i_0j}A_{ij},\forall i = 1,2,\dots,n,i \neq i_0
    \end{align*}
\end{itemize}

\begin{example}
  (2019S-1-1) 计算$
  \left|
    \begin{array}{cccc}
        2   & 3   & 4   & 5 \\
        -1  & 0   & 0   & 0 \\
        2   & -2  & 0   & 0 \\
        0   & 3   & -3  & 0 \\
    \end{array}
  \right|
  $
  的第一行所有元素的代数余子式之和。
\end{example}

\begin{solution}
    \begin{align*}
        A_{11} + A_{12} + A_{13} + A_{14} &= 
        \left|
        \begin{array}{cccc}
            1   & 1   & 1   & 1 \\
            -1  & 0   & 0   & 0 \\
            2   & -2  & 0   & 0 \\
            0   & 3   & -3  & 0 \\
        \end{array}
        \right| \\
        &= 6
    \end{align*}
\end{solution}

\begin{exercise}
    (2021S-1-1) 计算$A_{11} + M_{12} - M_{13}$,其中$D = 
    \left|
    \begin{array}{ccccc}
        2   & 1 & 1 & 1 & 1 \\
        1   & 2 & 1 & 1 & 1 \\
        1   & 1 & 2 & 1 & 1 \\
        1   & 1 & 1 & 2 & 1 \\
        1   & 1 & 1 & 1 & 2 \\
    \end{array}
    \right|
    $
    ,$A_{ij}$是元素$a_{ij}$的代数余子式,$M_{ij}$是元素$a_{ij}$的余子式。
\end{exercise}

\begin{exercise}
    设行列式
    $
    D=
    \left|
        \begin{array}{cccc}
            2 & 0 & -1 & 1 \\ 
            3 & 1 & 0 & 1 \\ 
            4 & 1 & 1 & 0 \\ 
            5 & -1 & 0 & a
        \end{array}
    \right|
    $
    ,$A_{ij}$表示元素$a_{ij}(i, j=1,2,3,4)$的代数余子式。若$A_{11}-A_{21}+A_{41}=4$,求$a$。
\end{exercise}

\section{分块行列式计算}

\begin{itemize}
    \item 
    $ 
    \left|
    \begin{array}{cc}
        A   & O \\
        C   & D \\
    \end{array}
    \right|
    =
    \left|
    \begin{array}{cc}
        A   & B \\
        O   & D \\
    \end{array}
    \right|
    = \left|A\right| \left|D\right|
    $
\end{itemize}

\begin{exercise}
    设$A$是$m$阶矩阵,$B$是$n$阶矩阵,且$\left|A\right|=a$,$\left|B\right|=b$,求
    $
    \left|
    \begin{array}{cc}
        O   & A \\
        B   & O \\
    \end{array}
	\right|
	.
    $
\end{exercise}

\section{多项式行列式计算}
关注各幂次前系数。

\begin{example}
    (2015W-2-(2)) \\
    (1) 计算$D_{n+1} = 
    \left|
    \begin{array}{ccccc}
        1         & 1         & \cdots & 1          & 1       \\
        x_1       & x_2       & \cdots & x_n        & x       \\
        x_1^2     & x_2^2     & \cdots & x_n^2      & x^2     \\
        \vdots    & \vdots    &        & \vdots     & \vdots  \\
        x_1^{n-2} & x_2^{n-2} & \cdots & x_n^{n-2}  & x^{n-2} \\
        x_1^{n-1} & x_2^{n-1} & \cdots & x_n^{n-1}  & x^{n-1} \\
        x_1^n     & x_2^n     & \cdots & x_n^n      & x^n     \\
    \end{array}
    \right|
	.
    $
    \\
    (2) 由(1)计算$C_n = 
    \left|
    \begin{array}{cccc}
        1         & 1         & \cdots & 1          \\
        x_1       & x_2       & \cdots & x_n        \\
        x_1^2     & x_2^2     & \cdots & x_n^2      \\
        \vdots    & \vdots    &        & \vdots     \\
        x_1^{n-2} & x_2^{n-2} & \cdots & x_n^{n-2}  \\
        x_1^n     & x_2^n     & \cdots & x_n^n      \\
    \end{array}
    \right|
	.
    $
\end{example}

\begin{solution}
    \ \\
    (2) 注意到$-C_n$为$D_{n+1}$的$x^{n-1}$项前系数,由Vieta定理,$C_n = \prod_{1 \le j < i \le n}(x_i - x_j) \sum_{i=1}^n x_i$
\end{solution}

\begin{exercise}
    (2019S-4改) 计算$f'(\pi)$与$f''(\pi)$,此处
    $$
    f(x) =
    \left|
    \begin{array}{ccc}
        a_1 & b_1 & a_1 x^2 + b_1 x + c_1 \\
        a_2 & b_2 & a_2 x^2 + b_2 x + c_2 \\
        a_3 & b_3 & a_3 x^2 + b_3 x + c_3 \\
    \end{array}
    \right|
	.
    $$
\end{exercise}

\begin{exercise}
    分析不恒为零的函数$f(x)=
    \left|
    \begin{array}{ccc}
        a_1 + x & b_1 + x & c_1 + x \\
        a_2 + x & b_2 + x & c_2 + x \\
        a_3 + x & b_3 + x & c_3 + x \\
    \end{array}
    \right|
    $
    的零点数量情况。
\end{exercise}

\begin{exercise}
    设$f(x)=
    \left|
    \begin{array}{ccccc}
        x + 1   & 2         & 3         & \cdots     & n         \\
        1       & x + 2     & 3         & \cdots     & n         \\
        1       & 2         & x + 3     & \cdots     & n         \\
        \vdots  & \vdots    & \vdots    &            & \vdots    \\
        1       & 2         & 3         & \cdots     & x + n     \\
    \end{array}
    \right|
    $
    ,求$f^{(n-1)}(0)$。
\end{exercise}

\chapter{矩阵基础性质、计算与初等变换}

\begin{introduction}[需要熟悉的知识点]
	\item 手算矩阵乘法
	\item 矩阵的逆、伴随、转置定义及性质
	\item 手算求逆,手算求伴随
	\item 矩阵的初等变换、初等矩阵,初等变换的矩阵表示,相抵关系,计算具体矩阵的秩,解线性方程组
	\item 分块矩阵处理与矩阵分块,简单矩阵分解
\end{introduction}

\section{基本性质与计算型}

\begin{itemize}
	\item $AB = 
        \left[
        \begin{matrix}
            \alpha_1^\mathrm{T} \\
            \alpha_2^\mathrm{T} \\
            \vdots \\
            \alpha_s^\mathrm{T} \\
        \end{matrix}
        \right]_{s \times n}
        [\beta_1,\beta_2,\cdots,\beta_m]_{n \times m}
        =
        \left[
        \begin{matrix}
            \alpha_1^\mathrm{T} \beta_1 & \alpha_1^\mathrm{T} \beta_2 & \cdots & \alpha_1^\mathrm{T} \beta_m \\
            \alpha_2^\mathrm{T} \beta_1 & \alpha_1^\mathrm{T} \beta_2 & \cdots & \alpha_1^\mathrm{T} \beta_m \\
            \vdots & \vdots & & \vdots \\
            \alpha_s^\mathrm{T} \beta_1 & \alpha_s^\mathrm{T} \beta_2 & \cdots & \alpha_s^\mathrm{T} \beta_m \\
        \end{matrix}
        \right]_{s \times m}
        $
	\item $A A^{-1} = A^{-1} A = I$,逆如果存在是唯一的
	\begin{itemize}
        \item $(A^{-1})^{-1} = A$
        \item $\left|A^{-1}\right| =  \left|A\right|^{-1}$
        \item 3阶及以下矩阵算逆用伴随,3阶以上矩阵算逆用初等行变换
    \end{itemize}
	\item $A A^* = A^* A = \left|A\right| I$,伴随是唯一的
	\begin{itemize}
        \item 一般没有$(A^*)^{*} = A$
        \item $\left|A^*\right| = \left|A\right|^{n-1}$
        \item $A^*$的$(i,j)$元是$A_{ji}$而非$A_{ij}$
    \end{itemize}
    \item $A^{-1} = \dfrac{1}{\left|A\right|} A^*$
    \begin{itemize}
        \item 看到逆和伴随,优先考虑乘一个原矩阵化简
        \item 最后不要忘记去除乘原矩阵的影响
    \end{itemize}
    \item 转置是唯一的
    \begin{itemize}
        \item $(A^\mathrm{T})^\mathrm{T} = A$
        \item $\left|A^\mathrm{T}\right| = \left|A\right|$
    \end{itemize}
    \item 初等行(列)变换等价于把对应的初等矩阵左(右)乘原矩阵
    \item 初等变换不改变秩,但可能改变行列式
\end{itemize}

\begin{example}
    (2015W-5) 一个方阵$A$称为幂零的,如果存在正整数$N$使得$A^N=0$.设A为幂零的,证明:\\
    (1) $B = a_1 A + a_2 A^2 + \dots + a_m A^m$也是幂零的.\\
    (2) 若$a_0 \neq 0$,则$C = a_0 E + a_1 A + a_2 A^2 + \dots + a_m A^m$是可逆矩阵,并利用$B$来表示$C^{-1}$.
\end{example}

\begin{proof}
    (1) 直接验证$B^N = O$,从而幂零。\\
    (2) 假设我们已经知道了$C$可逆,$C^{-1} = (a_0 E + B)^{-1}$,完全形式化地处理矩阵的逆。
    \begin{align*}
        (a_0 E + B)^{-1} &= \dfrac{E}{a_0 E + B} \\
        &= \dfrac{1}{a_0} \dfrac{E}{E + \dfrac{1}{a_0} B} \\
        &= \dfrac{1}{a_0} \sum_{k=0}^{\infty} \left(-\dfrac{1}{a_0} B\right)^k \\
        &= \dfrac{1}{a_0} \sum_{k=0}^{N-1} \left(-\dfrac{1}{a_0} B\right)^k
    \end{align*}
    这表明如果$C$可逆,那么$C^{-1} = \dfrac{1}{a_0} \sum_{k=0}^{N-1} \left(-\dfrac{1}{a_0} B\right)^k$,那么只需验证$C \dfrac{1}{a_0} \sum_{k=0}^{N-1} \left(-\dfrac{1}{a_0} B\right)^k = E$:
    \begin{align*}
        C \dfrac{1}{a_0} \sum_{k=0}^{N-1} \left(-\dfrac{1}{a_0} B\right)^k &= \dfrac{1}{a_0}  (a_0 E + B) \sum_{k=0}^{N-1} \left(-\dfrac{1}{a_0} B\right)^k \\
        &= \sum_{k=0}^{N-1} \left(-\dfrac{1}{a_0} B\right)^k - \sum_{k=1}^{N} \left(-\dfrac{1}{a_0} B\right)^k \\
        &= \sum_{k=0}^{N-1} \left(-\dfrac{1}{a_0} B\right)^k - \sum_{k=1}^{N-1} \left(-\dfrac{1}{a_0} B\right)^k \\
        &= E
    \end{align*}
    上面的过程请在草稿纸上完成。解题过程如下:\\
    断言:$C$可逆,并且$C^{-1} = \dfrac{1}{a_0} \sum_{k=0}^{N-1} \left(-\dfrac{1}{a_0} B\right)^k$。\\
    事实上,
    \begin{align*}
        C \dfrac{1}{a_0} \sum_{k=0}^{N-1} \left(-\dfrac{1}{a_0} B\right)^k &= \dfrac{1}{a_0}  (a_0 E + B) \sum_{k=0}^{N-1} \left(-\dfrac{1}{a_0} B\right)^k \\
        &= \sum_{k=0}^{N-1} \left(-\dfrac{1}{a_0} B\right)^k - \sum_{k=1}^{N} \left(-\dfrac{1}{a_0} B\right)^k \\
        &= \sum_{k=0}^{N-1} \left(-\dfrac{1}{a_0} B\right)^k - \sum_{k=1}^{N-1} \left(-\dfrac{1}{a_0} B\right)^k \\
        &= E
    \end{align*}
\end{proof}

\begin{exercise}
    (2015W-1-3) 设
    $
    A + B =
    \left[
    \begin{matrix}
        1 & 1 \\
        2 & 4 \\
    \end{matrix}
    \right]
    , A - B =
    \left[
    \begin{matrix}
        3 & 5 \\
        0 & 2 \\
    \end{matrix}
    \right]
    ,$
    求$A^2 - B^2$.
\end{exercise}

\begin{exercise}
    (2017S-1-1) 设$A$为3阶方阵,$\left|A\right| = \dfrac{1}{2}$,求$\left|(3 A)^{-1} - 2 A^*\right|$.
\end{exercise}

\begin{exercise}
    (2017S-1-4) 已知
    $
    A = 
    \left[
    \begin{matrix}
        -5 & 2 \\
        -12 & 5 \\
    \end{matrix}
    \right]
    ,
    P = 
    \left[
    \begin{matrix}
        1 & 2 \\
        3 & 4 \\
    \end{matrix}
    \right]
    $,求$P^{-1}AP$和$A^{-3} - A$.
\end{exercise}

\begin{exercise}
    (2017S-2) 若方阵满足$X^2=X$,则称$X$是幂等的。设$A$和$B$是同阶的幂等方阵,证明:$A+B$是幂等的当且仅当$AB=BA=O$。
\end{exercise}

\begin{exercise}
    (2018W-1-2) 设
    $
    A = 
    \left[
    \begin{matrix}
        0 & 1 &         &           &       \\
          & 0 & 2       &           &       \\
          &   & \ddots  & \ddots    &       \\
          &   &         & 0         & n-1   \\
        n &   &         &           & 0     \\
    \end{matrix}
    \right],
    B = 
    \left[
    \begin{matrix}
        2 & 1 \\
        5 & 3 \\
    \end{matrix}
    \right],
    C = 
    \left[
    \begin{matrix}
        A & O \\
        O & B \\
    \end{matrix}
    \right]
    $,其中$n \ge 2$,求$C^{-1}$.
\end{exercise}

\begin{exercise}
    (2018W-1-3) 设$A \in \mathbb{R}^{3 \times 3}$,$\left|A\right| \neq 0$,且$A_{ij} = 2 a_{ij}, i, j = 1,2,3$,其中$A_{ij}$为$a_{ij}$的代数余子式,求$\left| A^* \right|$.
\end{exercise}

\begin{exercise}
    (2019S-1-3) 已知4阶方阵$A$的伴随矩阵$A^* = 
    \left[
    \begin{matrix}
    0 & 0 & 3 & 2 \\
    0 & 0 & 0 & 3 \\
    2 & 1 & 0 & 0 \\
    1 & 2 & 0 & 0 \\
    \end{matrix}
    \right]
    $,求$A$.
\end{exercise}

\begin{exercise}
    (2019W-1-3) 已知$A^{-1} = 
    \left[
    \begin{matrix}
        2 & 1 & 4 \\
        1 & -2 & 3 \\
        -2 & 1 & -6 \\
    \end{matrix}
    \right]
    $,求$(E+A)^{-1}$.
\end{exercise}

\begin{exercise}
    (2019W-3) 设$n$阶方阵满足$(A^*)^* = O$,证明$\left|A\right| = 0$.
\end{exercise}

\begin{exercise}
    (2021S-1-4) 计算$(A^*)^*$,其中$A = 
    \left[
    \begin{matrix}
        2 & 1 & 1 \\
        1 & 2 & 1 \\
        1 & 1 & 2 \\
    \end{matrix}
    \right]
    $.
\end{exercise}

\begin{exercise}
    (2021S-5(1)) 设$A = (\alpha_1,\alpha_2,\dots,\alpha_n),B = A^{-1} = 
    \left[
    \begin{matrix}
        \beta_1^\mathrm{T} \\
        \beta_2^\mathrm{T} \\
        \vdots \\
        \beta_n^\mathrm{T} \\
    \end{matrix}
    \right]
    , C = \sum_{i=1}^k \alpha_i \beta_i^\mathrm{T} \quad (k<n)
    $. \\
    (1) 证明:$C^2 = C$.
\end{exercise}

\section{初等变换计算型}

\begin{itemize}
    \item 计算具体矩阵的秩
    \item 求解线性方程组
    \item 求解矩阵方程:$AX=B \iff A[x_1,x_2,\dots,x_n] = [\beta_1,\beta_2,\dots,\beta_n] \iff Ax_k = \beta_k,k=1,2,\dots,n$
\end{itemize}
解题方法相当统一且简单:对矩阵作初等行变换直至变为行阶梯型。
\textbf{多加练习!不要跳步!没有技巧!}

\begin{example}
    (2017S-4) 解带参数方程组
    $$
    \begin{cases}
        x_1 + (\lambda^2 + 1) x_2 + 2 x_3 = \lambda \\
        \lambda x_1 + \lambda x_2 + (2 \lambda + 1) x_3 = 0 \\
        x_1 + (2 \lambda + 1) x_2 + 2 x_3 = 2
    \end{cases}
    .
    $$
\end{example}

\begin{exercise}
    (2015W-1-2) 求$p$使得矩阵$
    A=
    \left[
    \begin{matrix}
        p & 4 & 10 & 1 \\
        1 & 7 & 15 & 3 \\
        2 & 2 & 0 & 3 \\
    \end{matrix}
    \right]
    $
    的秩$r(A)$最小,并求$r(A)$.
\end{exercise}

\begin{exercise}
    (2015W-4) 求过点$(-2,0),(-1,1),(1,-3),(t,1)$的三次多项式函数$y = a_3 x^3 + a_2 x^2 + a_1 x + a_0$,其中$t$为参数。
\end{exercise}

\begin{exercise}
    (2016W-5) 设$A = 
    \left[
    \begin{matrix}
        5 & 2 & 4 \\
        2 & 5 & -11 \\
        2 & 3 & -5 \\
    \end{matrix}
    \right]
    $. \\
    (1) 解方程组$A x = 0$. \\
    (2) 求满足$A^2 x = 0$但不满足$A x = 0$的$x$的集合. \\
\end{exercise}

\begin{exercise}
    (2017S-6) 设$A$和$X$为$n$阶方阵,且满足$AX = A + 2X$. \\
    (1) 证明:$AX = XA$. \\
    (2) 若$A = 
    \left[
    \begin{matrix}
        4 & 2 & 3 \\
        1 & 1 & 0 \\
        -1 & 2 & 3 \\
    \end{matrix}
    \right]
    $,求解矩阵方程$AX = A + 2X$.
\end{exercise}

\begin{exercise}
    (2018W-1-1) 设$A=
    \left[
    \begin{matrix}
        1 & 2 & 1 \\
        1 & 1 & 0 \\
        2 & 1 & -\lambda \\
    \end{matrix}
    \right]
    $
    经过多次初等行变换和列变换得到$B=
    \left[
    \begin{matrix}
        -5 & 17 & 6 \\
        -7 & 0 & 5 \\
        13 & 9 & -8 \\
    \end{matrix}
    \right]
    $
\end{exercise}

\begin{exercise}
    (2019S-1-2) 计算3阶方阵$X$使其满足$
    \left[
    \begin{matrix}
        1 & 1 & 1 \\
        1 & -1 & 0 \\
        0 & 0 & -1 \\
    \end{matrix}
    \right]
    X
    \left[
    \begin{matrix}
        1 & 1 & 1 \\
        1 & -1 & 0 \\
        0 & 0 & 1 \\
    \end{matrix}
    \right]
    =
    \left[
    \begin{matrix}
        1 & 1 & 0 \\
        1 & -1 & 0 \\
        0 & 2 & 2 \\
    \end{matrix}
    \right]
    $
\end{exercise}

\begin{exercise}
    (2019S-5) 给定矩阵$A = 
    \left[
    \begin{matrix}
        1 & -1 & -3 & -2 & -3 \\
        1 & 3 & 8 & -3 & 9 \\
        3 & 1 & 2 & -7 & 3 \\
    \end{matrix}
    \right]
    $ \\
    (1) 计算$r(A)$. \\
    (2) 计算线性方程组$Ax = 0$的基本解组. \\
    (3) 若$\eta = (1, -1, 0, 0, 2)^\mathrm{T}$是$Ax = b$的解,确定$b$并求$Ax = b$的通解.
\end{exercise}

\begin{exercise}
    (2019W-2) 设$A=
    \left[
    \begin{matrix}
        1 & -3 & 5 \\
        -2 & 1 & -3 \\
        -1 & -7 & 9 \\
    \end{matrix}
    \right]
    ,\beta=
    \left[
    \begin{matrix}
        4 \\
        -3 \\
        6 \\
    \end{matrix}
    \right]
    ,\gamma=
    \left[
    \begin{matrix}
        3 \\
        s \\
        2.4 \\
    \end{matrix}
    \right]
    $,其中$s$为参数。\\
    (1) 解方程组$Ax=\beta$. \\
    (2) 令$B=
    \left[
    \begin{matrix}
        A & \beta \\
        \gamma^\mathrm{T} & 3 \\
    \end{matrix}
    \right]
    $,解方程组$By = 0$.
\end{exercise}

\begin{exercise}
    (2020S-1-2) 设$A=
    \left[
    \begin{matrix}
        1 & -2 & 1 \\
        2 & 1 & 3 \\
        1 & -1 & 1 \\
    \end{matrix}
    \right]
    ,B=
    \left[
    \begin{matrix}
        -6 & 8 \\
        4 & 5 \\
        2 & 2 \\
    \end{matrix}
    \right]
    ,C=
    \left[
    \begin{matrix}
        1 & 3 \\
        2 & 2 \\
        3 & 1 \\
    \end{matrix}
    \right]
    $,求解矩阵方程$A(X - B) = C$.
\end{exercise}

\begin{exercise}
    (2020S-2) 解方程组
    $
    \begin{cases}
        2 x_1 + 3 x_2 - 5 x_3 + 4 x_4 = -11 \\
        x_1 + a x_2 + 2 x_3 - 7 x_4 = 7 \\
        3 x_1 - x_2 - 2 x_3 - 5 x_4 = 0
    \end{cases}
    $.
\end{exercise}

\begin{exercise}
    (2021S-1-5) 计算矩阵$X$使得$
    \left[
    \begin{matrix}
        2 & 1 & 0 \\
        1 & 2 & 1 \\
        0 & 1 & 2 \\
    \end{matrix}
    \right]
    X
    \left[
    \begin{matrix}
        1 & -1 & 0 & 0 \\
        0 & 1 & -1 & 0 \\
        0 & 0 & 1 & -1 \\
        0 & 0 & 0 & 1 \\
    \end{matrix}
    \right]
    =
    \left[
    \begin{matrix}
        1 & 0 & 0 & 0 \\
        -1 & 1 & 0 & 0 \\
        0 & -1 & 1 & 0 \\
    \end{matrix}
    \right]
    $.
\end{exercise}

\begin{exercise}
    (2021S-5(2)) 设$A = (\alpha_1,\alpha_2,\dots,\alpha_n),B = A^{-1} = 
    \left[
    \begin{matrix}
        \beta_1^\mathrm{T} \\
        \beta_2^\mathrm{T} \\
        \vdots \\
        \beta_n^\mathrm{T} \\
    \end{matrix}
    \right]
    , C = \sum_{i=1}^k \alpha_i \beta_i^\mathrm{T} \quad (k<n)
    $. \\
    (2) 写出$Cx = 0$的一个基础解系。
\end{exercise}

\section{矩阵分解型}

\begin{itemize}
    \item 求$A^n$
    \begin{itemize}
        \item 低秩矩阵将其分解为低维矩阵(向量)之积后利用结合律
        \item 幂零矩阵,如$
        \left[
        \begin{matrix}
            0 & 1 &         & \\
              & 0 & \ddots &  \\
              &   & \ddots & 1  \\
              &   &         & 0 \\
        \end{matrix}
        \right]
        $
    \end{itemize}
    \item 矩阵的每一列被写为向量组线性表出形式
\end{itemize}

\begin{example}
    (2016W-1-1) 已知矩阵
    $
    A = 
    \left[
    \begin{matrix}
        3 & 6 & -3 \\
        -1 & -2 & 1 \\
        2 & 4 & -2  \\
    \end{matrix}
    \right]
    $,求$A^n$.
\end{example}

\begin{solution}
    $
    A = 
    \left[
    \begin{matrix}
        3 \\
        -1 \\
        2 \\
    \end{matrix}
    \right]
    \left[
    \begin{matrix}
        1 & 2 & -1 \\
    \end{matrix}
    \right]
    = \alpha \beta^{\mathrm{T}}
    $,当$n \ge 2$时,$A^n = \alpha (\beta^{\mathrm{T}} \alpha)^{n-1} \beta^{\mathrm{T}} = \alpha (-1)^{n-1} \beta^{\mathrm{T}} = (-1)^{n-1} \alpha \beta^{\mathrm{T}} = (-1)^{n-1} A$,故$A^n = (-1)^{n-1} A, n \ge 1$.
\end{solution}

\begin{exercise}
    (2021S-1-2) 计算$A^{2021}$,其中$A =  
    \left[
    \begin{matrix}
        -1 & 1 & -2 & -1 \\
        1 & -1 & 2 & 1 \\
        0 & 0 & 0 & 0 \\
        2 & -2 & 4 & 2 \\
    \end{matrix}
    \right]
    $.
\end{exercise}

\begin{example}
    (2016W-1-2) 已知矩阵$A = (\alpha_1,\alpha_2,\alpha_3) \in \mathbb{R}^{3 \times 3}$,$\left|A\right|=3$,矩阵$B = (2 \alpha_1 - \alpha_2 + 2 \alpha_3, \alpha_1 - 3 \alpha_2 + \alpha_3, \alpha_1 + 2 \alpha_2 - \alpha_3)$,计算$\left|B\right|$.
\end{example}

\begin{solution}
    $
    \left|B\right| = \left|A\right|
    \left|
    \begin{matrix}
        2 & 1 & 1 \\
        -1 & -3 & 2 \\
        2 & 1 & -1 \\
    \end{matrix}
    \right|
    = 30.
    $
\end{solution}

\begin{exercise}
    (2019W-1-4) 设$A = (\alpha_1, \alpha_2, \alpha_3),B = (-3 \alpha_2 + \alpha_3, \alpha_1 - \alpha_2 + 2 \alpha_3, -2 \alpha_1 + \alpha_2 - \alpha_3), \left|B\right| = 16$,求$\left|A+B\right|$.
\end{exercise}

\chapter{矩阵的秩、向量线性相关性与线性方程组解的结构}

\begin{introduction}[需要熟悉的知识点]
    \item 线性相关,线性无关,线性表出,向量组等价,向量组的秩
	\item 矩阵行秩、列秩,秩的行秩列秩定义,秩的子式定义
	\item Sylvester秩不等式
	\item 矩阵加法、乘法、组合对秩的影响
	\item 线性方程组解的结构
	\item 线性方程组无解、有解、有唯一解、有无穷多解条件
	\item 维数定理
\end{introduction}

\section{秩的相关性质}

\begin{itemize}
    \item Sylvester秩不等式:$r(A) + r(B) - n \le r(AB) \le \min(r(A),r(B))$
    \begin{itemize}
        \item 当$AB = O$时,$r(A) + r(B) \le n$
        \item 如果$A$为可逆方阵,$r(AB) = r(B)$,$B$为可逆方阵时同理
    \end{itemize}
    \item $r(A + B) \le r(A) + r(B)$
    \item $r(A,B) \le r(A) + r(B), 
        r\left(\left[
            \begin{matrix}
                A \\
                B \\
            \end{matrix}
            \right]\right) \le r(A) + r(B)$
    \item $r\left(\left[
        \begin{matrix}
            A & O \\
            O & B \\
        \end{matrix}
        \right]\right) = r(A) + r(B)$
\end{itemize}

\begin{example}
    (2015W-6) 若$n$阶方阵$A$满足$A^2=A$,则称$A$为幂等矩阵。设$A$为幂等矩阵,证明:\\
    (1) $E - A$也为幂等矩阵。\\
    (2) $r(A) + r(E - A) = n$.
\end{example}

\begin{proof}
    (1) $(E - A)^2 = E - 2A + A^2 = E - 2A + A = E - A$. \\
    (2) 注意到$A(E - A) = O$,由Sylvester秩不等式,$ n = r(A + E - A) \le r(A) + r(E - A) \le n$即得$r(A) + r(E - A) = n$.
\end{proof}

\begin{exercise}
    (2017S-1-2) 设$A$为$n$阶方阵且满足$A^2 = -A$,证明:$r(A) + r(E + A) = n$.
\end{exercise}

\begin{exercise}
    (2018W-1-4) 设$A = M N^\mathrm{T}$,其中$M,N \in \mathbb{R}^{n \times r}(r \le n)$,$\left|N^\mathrm{T} M\right| \neq 0$。证明:$r(A^2) = r(A)$。
\end{exercise}

\begin{exercise}
    (2019S-1-4改) 对于$A = \{ \alpha_1,\alpha_2,\dots,\alpha_{100}\},B = \{\beta_1,\beta_2,\dots,\beta_{20}\}$,若$r(A) = 7$,给出$r(A \cup B)$的取值范围。
\end{exercise}

\begin{exercise}
    (2019W-3) 设$n$阶方阵满足$(A^*)^* = O$,证明$\left|A\right| = 0$.
\end{exercise}

\begin{exercise}
    (2020S-6) 设$A$为$n$阶方阵,$r(A) = n-1$,证明:$A^* = \alpha \beta^\mathrm{T}$,其中$\alpha,\beta$为$n$维列向量,且有$A \alpha = A^\mathrm{T} \beta = 0$。 
\end{exercise}

\section{线性方程组解的结构}

\begin{itemize}
    \item 抓住$r(A)$与$r(A | b)$
\end{itemize}

\begin{exercise}
    (2019S-5) 设矩阵$A=
    \left[
    \begin{matrix}
        1 & -1 & -3 & -2 & -3 \\
        1 & 3 & 8 & -3 & 9 \\
        3 & 1 & 2 & -7 & 3 \\
    \end{matrix}
    \right]
    $.\\
    (1) 计算$r(A)$。\\
    (2) 计算线性方程组$Ax = 0$的基本解组。\\
    (3) 假定$\eta = (1,-1,0,0,2)^\mathrm{T}$是$Ax=b$的解,确定$b$并计算$Ax=b$的通解。
\end{exercise}

\section{秩与线性方程组解结构的相互转化}

\begin{itemize}
    \item 维数定理:对于齐次线性方程组$Ax = 0$,$x \in \mathbb{R}^n$,若$r(A) = r$,方程的一个基础解系中有$n-r$个向量。
\end{itemize}

\begin{example}
    (2016W-3) 设$A_1 x = b_1$和$A_2 x = b_2$是两个非齐次线性方程组,其中$A_1 \in \mathbb{R}^{m \times n}, A_2 \in \mathbb{R}^{k \times n}$.如果这两个方程组同解,证明$A_1$和$A_2$的行向量组等价。
\end{example}

\begin{proof}
    $A_1 x = b_1,A_2 x = b_2,
    \left[
    \begin{matrix}
        A_1 \\
        A_2 \\
    \end{matrix}
    \right]x=
    \left[
    \begin{matrix}
        b_1 \\
        b_2 \\
    \end{matrix}
    \right]
    $均同解,这表明$r(A_1) = r(A_2) = r\left(
    \left[
    \begin{matrix}
        A_1 \\
        A_2 \\
    \end{matrix}
    \right]
    \right)
    $,得证。
\end{proof}

\begin{example}
    (2019S-7改, 2021S-4改) 设$A$为$m \times n$阶实矩阵,$b$为$m$维实向量。证明:\\
    (1) $A^\mathrm{T} A x = 0$和$A x = 0$同解。\\
    (2) $r(A) = r(A^\mathrm{T} A) = r(A A^\mathrm{T})$。\\
    (3) $A^\mathrm{T} A x = A^\mathrm{T}b$恒有解。\\
    (4) 如果$A x = b$有解,那么$A^\mathrm{T} A x = A^\mathrm{T} b$和$A x = b$同解。
\end{example}

\begin{proof}
    (1) 显然当$x$是$Ax=0$的解时,$x$是$A^\mathrm{T} A x = 0$的解。反之,设$x$是$A^\mathrm{T} A x = 0$的解,那么$(A x)^\mathrm{T} A x = 0$,这表明$A x = 0$,得证。\\
    (2) 由(1),$r(A) = r(A^\mathrm{T} A)$。又$r(A) = r(A^\mathrm{T})$,$r(A^\mathrm{T} A) = r(A) = r(A^\mathrm{T}) = r((A^\mathrm{T})^\mathrm{T} A^\mathrm{T}) = r(A A^\mathrm{T})$ \\
    (3) $r(A^\mathrm{T} A | A^\mathrm{T} b) = r(A^\mathrm{T}) = r(A^\mathrm{T} A)$ \\
    (4) 显然当$x$是$Ax=b$的解时,$x$是$A^\mathrm{T} A x = A^\mathrm{T} b$的解。反之,设$x$是$A^\mathrm{T} A x = A^\mathrm{T} b$的解,$x_0$是$Ax=b$的解,由上,$x_0$是$A^\mathrm{T} A x = A^\mathrm{T} b$的解。下验证$Ax = b$成立,只需证明$Ax - b = 0$即可。\\
    $A^\mathrm{T} (Ax - b) = A^\mathrm{T} A(x - x_0) = 0$,那么$(A(x - x_0))^\mathrm{T} A(x - x_0) = 0$,这表明$A(x - x_0) = 0,Ax = Ax_0 = b$,从而$x$是$Ax=b$的解,得证。
\end{proof}

\begin{remark}
    上题结论请掌握,并熟悉证明过程。
\end{remark}

\begin{exercise}
    (2015W-1-5) 若5元方程组$Ax=b,b \neq 0$有解$\xi_1=(1,1,1,1,1)^\mathrm{T},\xi_2=(1,2,3,4,5)^\mathrm{T},\xi_3=(1,0,-3,-2,-3)^\mathrm{T}$,且$r(A)=3$,求方程组的通解。
\end{exercise}

\begin{exercise}
    (2016W-1-4) 设$A \in \mathbb{R}^{m \times n} (m>n)$,$r(A)=n$,证明:存在矩阵$P \in \mathbb{R}^{n \times m}$使得$PA = E_n$.
\end{exercise}

\begin{exercise}
    (2016W-3改) 设$A_1 x = b_1$和$A_2 x = b_2$是两个非齐次线性方程组,其中$A_1 \in \mathbb{R}^{m \times n}, A_2 \in \mathbb{R}^{k \times n}$.如果这两个方程组同解,证明$[A_1 | b_1]$和$[A_2 | b_2]$的行向量组等价。
\end{exercise}

\begin{exercise}
    (2017S-1-5) 设$n$阶方阵$A$的秩为$r<n$,证明存在秩为$n-r$的$n$阶方阵$B$使得$AB=O$.
\end{exercise}

\begin{exercise}
    (2017S-3) 证明方程组
    $$
    \begin{cases}
        x_1 - x_2 &= a_1 \\
        x_2 - x_3 &= a_2 \\
        &\vdots \\
        x_{n-1} - x_n &= a_{n-1} \\
        x_n - x_1 &= a_n \\
    \end{cases}
    $$
    有解当且仅当$\sum_{i=1}^n a_i = 0$,并在有解的情况下,求方程组的解集。
\end{exercise}

\begin{exercise}
    (2019S-1-5) 已知线性方程组$Ax=b$的三个特解为$\alpha_1 = (1,-2,3)^\mathrm{T},\alpha_2 = (0,-1,-2)^\mathrm{T},\alpha_3 = (-4,2,1)^\mathrm{T},r(A) = 1$,写出$Ax=b$的通解。
\end{exercise}

\begin{exercise}
    (2019W-1-5) 设$A,B$为$m \times n$阶矩阵,证明$r(A A^\mathrm{T} + B B^\mathrm{T}) = r(A,B)$.
\end{exercise}

\begin{exercise}
    (2021S-1-6) 设$A$为3阶方阵,$r(A)=1$,$\alpha_1,\alpha_2,\alpha_3$是$Ax=b$的三个解向量,$\alpha_1 + \alpha_2 = (1,-1,3)^\mathrm{T},\alpha_2 + \alpha_3 = (0,-2,1)^\mathrm{T},\alpha_3 + \alpha_1 = (3,0,4)^\mathrm{T}$,求$Ax=b$的通解。
\end{exercise}

\begin{exercise}
    (2021S-6) 设$A$为$m \times n$阶矩阵,$B$为$n \times k$阶矩阵,$AB=O$,$r(A) + r(B) = n$,$B$与$C=(\gamma_1,\gamma_2,\dots,\gamma_k)$列等价,证明$\gamma_1,\gamma_2,\dots,\gamma_k$的极大线性无关组可以成为$Ax=0$的一个基础解系。
\end{exercise}

\section{向量线性相关性与线性方程组解的相互转化}

\begin{exercise}
    (2016W-2) 设$\alpha_1=
    \left[
    \begin{matrix}
        3 \\
        1 \\
        2 \\
        -1 \\
    \end{matrix}
    \right],\alpha_2=
    \left[
    \begin{matrix}
        1 \\
        -1 \\
        4 \\
        2 \\
    \end{matrix}
    \right],\alpha_3=
    \left[
    \begin{matrix}
        7 \\
        5 \\
        -2 \\
        -7 \\
    \end{matrix}
    \right],\alpha_4=
    \left[
    \begin{matrix}
        5 \\
        -1 \\
        10 \\
        3 \\
    \end{matrix}
    \right],\alpha_5=
    \left[
    \begin{matrix}
        -1 \\
        2 \\
        1 \\
        0 \\
    \end{matrix}
    \right]
    $. \\
    (1) 求$\alpha_1,\alpha_2,\alpha_3,\alpha_4,\alpha_5$的一个极大线性无关组,并用该极大线性无关组表出其他向量。\\
    (2) 若有向量$\beta = (1,0,-1,-1)^\mathrm{T}$,向量组$\alpha_3,\alpha_4,\beta$是否与$\alpha_1,\alpha_3,\alpha_3,\alpha_4,\alpha_5$等价?
\end{exercise}

\begin{exercise}
    (2018W-2) 设有向量组
    $$
    \alpha_1=
    \left[
    \begin{matrix}
        2 \\
        -2 \\
        -1 \\
        4 \\
    \end{matrix}
    \right],
    \alpha_2=
    \left[
    \begin{matrix}
        1 \\
        -2 \\
        0 \\
        1 \\
    \end{matrix}
    \right],
    \alpha_3=
    \left[
    \begin{matrix}
        1 \\
        -4 \\
        1 \\
        -1 \\
    \end{matrix}
    \right],
    \alpha_4=
    \left[
    \begin{matrix}
        1 \\
        0 \\
        2 \\
        3 \\
    \end{matrix}
    \right],
    \alpha_5=
    \left[
    \begin{matrix}
        2 \\
        1 \\
        2 \\
        7 \\
    \end{matrix}
    \right]
    $$
    (1) 求一个极大线性无关组,并用其表示其余向量。\\
    (2) 找出在$e_1,e_2,e_3,e_4$中所有不能被向量组$\alpha_1,\alpha_2,\alpha_3,\alpha_4,\alpha_5$线性表出的向量。
\end{exercise}

\begin{exercise}
    (2019W-4) 设向量组$A: \alpha_1=
    \left[
    \begin{matrix}
        2 \\
        4 \\
        3 \\
        1 \\
    \end{matrix}
    \right],\alpha_2=
    \left[
    \begin{matrix}
        4 \\
        8 \\
        6 \\
        2 \\
    \end{matrix}
    \right],\alpha_3=
    \left[
    \begin{matrix}
        1 \\
        3 \\
        1 \\
        2 \\
    \end{matrix}
    \right]
    $
    和$B: \beta_1=
    \left[
    \begin{matrix}
        2 \\
        4 \\
        3 \\
        1 \\
    \end{matrix}
    \right],\alpha_2=
    \left[
    \begin{matrix}
        4 \\
        8 \\
        6 \\
        2 \\
    \end{matrix}
    \right],\alpha_3=
    \left[
    \begin{matrix}
        1 \\
        3 \\
        1 \\
        2 \\
    \end{matrix}
    \right]
    $. \\
    (1) 分别求向量组$A$和$B$的一个极大线性无关组。\\
    (2) 求向量$\gamma$使得向量组$\alpha_1,\alpha_2,\alpha_3,\gamma$和$\beta_1,\beta_2,\beta_3,\gamma$等价。
\end{exercise}

\begin{exercise}
    (2020S-4) 设有向量组
    $$
    \alpha_1=
    \left[
    \begin{matrix}
        1 \\
        8 \\
        -1 \\
        4 \\
    \end{matrix}
    \right],
    \alpha_2=
    \left[
    \begin{matrix}
        3 \\
        0 \\
        -5 \\
        4 \\
    \end{matrix}
    \right],
    \alpha_3=
    \left[
    \begin{matrix}
        0 \\
        12 \\
        1 \\
        4 \\
    \end{matrix}
    \right],
    \alpha_4=
    \left[
    \begin{matrix}
        2 \\
        6 \\
        -2 \\
        4 \\
    \end{matrix}
    \right],
    \alpha_5=
    \left[
    \begin{matrix}
        -2 \\
        -4 \\
        3 \\
        -4 \\
    \end{matrix}
    \right]
    $$
    (1) 求一个极大线性无关组,并用其表示其余向量。\\
    (2) 在$\alpha_1,\alpha_2,\alpha_3,\alpha_4,\alpha_5$中去除一个向量,使得去除后向量组的秩减小。
\end{exercise}

\end{document}
